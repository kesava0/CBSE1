\documentclass[12pt,-letter paper]{article}
%\usepackage[left=1.5in,right=1in,top=1in,bottom=1in]{geometry}
%\usepackage[left=1.5in,right=1in]{geometry}
%\usepackage{geometry}
%\makeatletter%
%\textheight     243.5mm
%\textwidth      183.0mm
%\textwidth=31pc%
%\textheight=48pc
\usepackage{lipsum}% this package is included to get dummy paragraphs for sample purpose.
\usepackage{ulem}
\usepackage{alltt}
\usepackage{tfrupee}
\usepackage[anticlockwise,figuresright]{rotating}
\usepackage{pstricks}
\usepackage{wrapfig}
\usepackage{pstcol,pst-grad}
 \usepackage{bm}
\usepackage{enumitem}
\usepackage{listings}
    \usepackage{color}                                            %%
    \usepackage{array}                                            %%
    \usepackage{longtable}                                        %%
    \usepackage{calc}                                             %%
    \usepackage{multirow}                                         %%
    \usepackage{hhline}                                           %%
    \usepackage{ifthen}                                           %%
  %optionally (for landscape tables embedded in another document): %%
    \usepackage{lscape}     
    \usepackage{gensymb}     
    \usepackage{tabularx}
\usepackage{ifthen}%
\usepackage{amsmath}%
\usepackage{color}%
\usepackage{float}%
\usepackage{graphicx}%
%\usepackage[right]{showlabels}%
\usepackage{boites}%
\usepackage{boites_exemples}%
\usepackage{graphicx,pstricks}
%\usepackage{enumerate}%
\usepackage{latexsym}
\usepackage[fleqn]{mathtools}
\usepackage{amssymb}
\usepackage{amssymb,amsfonts,amsthm}
\usepackage{mathrsfs,makeidx,listings,verbatim,moreverb}
%%\usepackage{amsthm,mathrsfs,makeidx,listings,verbatim,moreverb}
%\let\eqref\ref%  updated on 20th April 2017

\usepackage{hyperref}%
%\usepackage[dvips]{hyperref}%
\hypersetup{bookmarksopen=false}%
\usepackage{breakurl}%
\usepackage{tkz-euclide} % loads  TikZ and tkz-base
\DeclarePairedDelimiter\abs{\lvert}{\rvert}

\newcommand{\solution}{\noindent \textbf{Solution: }}
\providecommand{\mbf}{\mathbf}
\providecommand{\rank}{\text{rank}}
%\providecommand{\pr}[1]{\ensuremath{\Pr\left(#1\right)}}
\providecommand{\qfunc}[1]{\ensuremath{Q\left(#1\right)}}
\providecommand{\sbrak}[1]{\ensuremath{{}\left[#1\right]}}
\providecommand{\lsbrak}[1]{\ensuremath{{}\left[#1\right.}}
\providecommand{\rsbrak}[1]{\ensuremath{{}\left.#1\right]}}
\providecommand{\brak}[1]{\ensuremath{\left(#1\right)}}
\providecommand{\lbrak}[1]{\ensuremath{\left(#1\right.}}
\providecommand{\rbrak}[1]{\ensuremath{\left.#1\right)}}
\providecommand{\cbrak}[1]{\ensuremath{\left\{#1\right\}}}
\providecommand{\lcbrak}[1]{\ensuremath{\left\{#1\right.}}
\providecommand{\rcbrak}[1]{\ensuremath{\left.#1\right\}}}
\newenvironment{amatrix}[1]{%
  \left(\begin{array}{@{}*{#1}{c}|c@{}}
}{%
  \end{array}\right)
}
\theoremstyle{remark}
\newtheorem{rem}{Remark}
\newtheorem{theorem}{Theorem}[section]
\newtheorem{problem}{Problem}
\newtheorem{proposition}{Proposition}[section]
\newtheorem{lemma}{Lemma}[section]
\newtheorem{corollary}[theorem]{Corollary}
\newtheorem{example}{Example}[section]
\newtheorem{definition}[problem]{Definition}
\newcommand{\sgn}{\mathop{\mathrm{sgn}}}
%\providecommand{\abs}[1]{\left\vert#1\right\vert}
%\providecommand{\res}[1]{\Res\displaylimits_{#1}} 
%\providecommand{\norm}[1]{\left\lVert#1\right\rVert}
%\providecommand{\norm}[1]{\lVert#1\rVert}
\providecommand{\mtx}[1]{\mathbf{#1}}
%\providecommand{\mean}[1]{E\left[ #1 \right]}
\providecommand{\fourier}{\overset{\mathcal{F}}{ \rightleftharpoons}}
%\providecommand{\hilbert}{\overset{\mathcal{H}}{ \rightleftharpoons}}
\providecommand{\system}{\overset{\mathcal{H}}{ \longleftrightarrow}}
	%\newcommand{\solution}[2]{\textbf{Solution:}{#1}}
%\newcommand{\solution}{\noindent \textbf{Solution: }}
\newcommand{\cosec}{\,\text{cosec}\,}
\providecommand{\dec}[2]{\ensuremath{\overset{#1}{\underset{#2}{\gtrless}}}}
\newcommand{\myvec}[1]{\ensuremath{\begin{pmatrix}#1\end{pmatrix}}}
\newcommand{\myaugvec}[2]{\ensuremath{\begin{amatrix}{#1}#2\end{amatrix}}}
\newcommand{\mydet}[1]{\ensuremath{\begin{vmatrix}#1\end{vmatrix}}}
\newcommand\figref{Fig.~\ref}
\newcommand\appref{Appendix~\ref}
\newcommand\tabref{Table~\ref}
\newcommand{\romanNumeral}[1]{\uppercase\expandafter{\romannumeral#1}}
%\newcommand{\pr}[1]{\mathbb{P}(#1)}
%\numberwithin{equation}{section}
%\numberwithin{equation}{subsection}
%\numberwithin{problem}{section}
%\numberwithin{definition}{section}
%\makeatletter
%\@addtoreset{figure}{problem}
%\makeatother

%\let\StandardTheFigure\thefigure
\let\vec\mathbf
\def\inputGnumericTable{}                                 %%
%New macro definitions
\newcounter{matchleft}\newcounter{matchright}

\newenvironment{matchtabular}{%
  \setcounter{matchleft}{0}%
  \setcounter{matchright}{0}%
  \tabularx{\textwidth}{%
    >{\leavevmode\hbox to 1.5em{\stepcounter{matchleft}\arabic{matchleft}}}X%
    >{\leavevmode\hbox to 1.5em{\stepcounter{matchright}\alph{matchright}}}X%
    }%
}{\end tabularx}

\title{\textbf{CBSE MATHS PAPER}}
\author{\textbf{KESAVA}}
\date{\textbf{\today}}

\begin{document}
\maketitle{\textbf{Questions}}

\begin{enumerate}
\section{Matrices}
	\item If A is a square matrix of order 3 with $\begin{vmatrix}A \end{vmatrix}$=$4$,then write the value of $\begin{vmatrix}-2A \end{vmatrix}$.	
 \item A =
$\begin{bmatrix}$ 
$3 & 9 & 0 \\ 1 & 8 & -2\\ 7 & 5 & 4$
$\end{bmatrix}$
and  B=
$\begin{bmatrix} $
$4 & 0 & 2 \\ 7 & 1 & 4 \\ 2 & 2 & 6$ 
$\end{bmatrix}$, 
then find the matrix B'A'.
\item Using properties of determinants,prove that
	  $\begin{bmatrix}
		  b+c & a & c \\  b & c+a & b \\ c & c & a+b\\
	  \end{bmatrix}$=$4abc$.
  \item If A = $\begin{bmatrix}
		  1&1&1 \\ 0&1&3 \\ 1&-2&1
  \end{bmatrix}$,find $A^{-1}$.
Hence, solve the system of equations:
    $x+y+z=6 ,\\
    y+3z=11 ,\\
    x-2y+z=0$.
\item Find the inverse of the following matrix, using elementary
transformations: 
	A=$\begin{bmatrix}
		2 & 3 & 1\\2 & 4 & 1\\3 & 7 & 2 
	\end{bmatrix}$.
\section{Integrations}
  \item $\int_a^b{\frac{logx}{x}dx}$.
  \item $\int{\frac{sin(x-a)}{sin(x+a)}dx}$.
  \item $\int{(logx)^{2}}$.
\item Find the value of x,if $\tan[\sec^{-1}(\frac{1}{x})]$ =$\sin(tan^{-1}2)$,$x>0$.
\item $\int{\frac{sin(2x)}{(sin^{2}x+1)(sin^{2}x+3)}dx}$.
	\item Prove that $\int_a^b{ f(x) dx}$=$\int_a^b{ f(a+b-x)dx}$ and hence evaluate.
	\item $\int_\frac{\pi}{6}^\frac{\pi}{3}{\frac{dx}{1+\sqrt{\tan{x}}}}$.
	\item Find  $\int_4^1{ (1+x+e^{2x})}$ evaluate aslimits as sum.
	\item Find the area of the region ${(x, y) : 0 \leq y \leq x^{2}, 0 \leq y \leq x + 2, – 1 \leq x \leq 3}$.
		\section{Differentiations}
		\item Find the Integral factor of the differential equation: $x\frac{dy}{dx}-2y$=$2x^{2}$.
	\item If $y$=$\sin^{-1}x+\cos^{-1}x$,find $\frac{dy}{dx}$.
		\item If $e^{y}(x+1)$=$1$, Then shows that $\frac{d^{2}y}{dx^{2}}$=$(\frac{dy}{dx})^{2}$.
			\item Find $\frac{dy}{dx}$,if y=$\sin^{-1}[\frac{2^{x+1}}{1+4^{x}}]$.
	    \item If y=$(\sec^{-1}x)^{2}$,$x>0$,show that $x^{2}(x^{2}-1)\frac{d^{2}y}{dx^{2}}+(2x^{3}-x) \frac{dy}{dx}-2$= 0.
	\item Solve the differential equation: $\frac{dy}{dx}$= $\frac{x+y}{x-y}$.
	\item Solve the differential equation: $(1+x^{2})dy+2xydy$=$\cot{x}dx$.
	\item Form the differential equation representing the family of curves $y^{2}$= $m(a^{2}–x^{2})$ by eliminating the arbitrary constants ‘m’ and ‘a’.

		\section{Geometry}
		\item Find the values of p for which the following lines are perpendicular
  $\frac{1-x}{3}=\frac{2y-14}{2p}=\frac{z-3}{2};\frac{1-x}{3p}=\frac{y-5}{1}=\frac{6-z}{5}$.
  \item Find the direction cosines of the line joining  the points P($4,3,-5$) and Q($-2,1,-8$).
   \item Find the value of $\lambda$ for which the following lines are per    pendicular to each other: $\frac{x-5}{5\lambda+2}= \frac{2-y}{5} = \frac{1-z}{-1}; \frac{x}{1}=\frac{y+\frac{1}{2}}{2\lambda}=\frac{z-1}{3}$. Hence,find whether the lines insert or not.
	  \item  Show that the height of a cylinder, which is open at the top, ha    ving a given surface area and greatest volume, is equal to the radius of its base.
	  \section{Vectors}
  \item Find a unit vector perpendicular to both vectors $\overrightarrow{a}$,$\overrightarrow{b}$,where $\vec{a}$=$\hat{i}-7\hat{j}+7\hat{k}$ and $\vec{b}$=$3\hat{i}-2\hat{j}+2\hat{k}$.
  \item Shows that the vector $\hat{i}-2\hat{j}+3\hat{k}$ and $\hat{i}-3\hat{j}+5\hat{k}$ are coplanar.
	  \item Let  $ \overrightarrow a ,\overrightarrow b$ and $\overrightarrow c $ be three vectors such that $\mydet{\overrightarrow{a}}  = 1 ,\mydet{\overrightarrow{b}}= 2$ and $\overrightarrow{|c|} =3$ .If the projection of $\overrightarrow b $ along $\overrightarrow a $ is equal to the projection of $\overrightarrow c $ along $ \overrightarrow a $; and $\overrightarrow b,\overrightarrow c $ are perpendicular to each other, then find $\abs{3\overrightarrow a-2\overrightarrow b +2\overrightarrow c}$.
		  \item Find the vector and Cartesian equations of the plane passing through the points $\brak{2,5,-3},\brak{-2,-3,5},$ and $\brak{5,3,-3}$.Also, find the point of intersection of this plane with the line passing through points $\brak{3,1,5} $ and $\brak{-1,-3,-1}.$
			  \item Find the equation of the plane passing through the intersection of the planes $\overrightarrow r.\brak{\hat{i}+\hat{j}+\hat{k}}=1 $ and$\overrightarrow  r .2\hat{i}+3\hat{j}-\hat{k}+4=0 $ and parallel to $x-axis$. Hence, find the distance of the plane from $x-axis$.
	  \section{Probability}
  \item Let X be a random variable which assumes values x1,x2,x3,x4 such that $2P(X=x1)=3p(X=x2)=p(X=x3)=5P(X=x4)$. Find the probability distribution of X.
	  \section{Functions}
  \item If $ $ is defined on the set of all real numbers by$ *$ :$a * b= \sqrt{a^{2} + b^{2}}$,find the identity element,if it exists in with respect to $$.
  \item Find the intervals in which the function f given by $f(x)$=$4x^{3}–6x^{2}–72x+30$ is (a) strictly increasing, (b) strictly decreasing.
  \item  Show that the relation R on the set Z of integers,given by R=\{(a, b):$ 2 \hspace{2mm} divides \hspace{2mm}(a–b)$\}. is an equivalence relation.
  \item If f(x)=$\frac{4x+3}{6x-4}$, $x\neq\frac{2}{3}$,show that fof(x)=x for all $x\neq\frac{2}{3}$. Also, find the inverse of f.
	  \item Mother,father and son line up at random for a family phot. If A and B are two events given by A=Son on one end,B=Father in the middle,find P($\frac{b}{a}$).
	  \item Find the intervals in which the function f given by $f(x)$ =$4x^{3}–6x^{2}–72x+30$ is (a) strictly increasing, (b) strictly decreasing.
		  \section{Optimization}
		  \item A company manufactures two types of novelty souvenirs made of plywood. Souvenirs of type $A$ require $5$ minutes each for cutting and $10$ minutes each for assembling. Souvenirs of type $B$ require $8$ minutes each for cutting and $8$ minutes each for assembling. There are $3$ hours and $20$ minutes available for cutting and $4$ hours available for assembling. The profit is \rupee~$50$ each for type $A$ and \rupee~$60$ each for type $B$ souvenirs. How many souvenirs of each type should the company manufacture in order to maximize profit ? Formulate the above $LPP$ and solve it graphically and also find the maximum profit.

\end{enumerate}

  \end{document}